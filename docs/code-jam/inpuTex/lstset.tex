% C++ Definition

\newcommand{\lstsetCpp}[1]
{ 
\lstset{caption={#1}, 
	language=[ANSI]C++,
	breaklines=true, 
        frame = tb, 
        framerule = 0.25pt, 
        float, 
        fontadjust, 
        backgroundcolor={\color{listlightgray}}, 
        basicstyle = {\ttfamily\footnotesize}, 
        keywordstyle = {\ttfamily\color{listkeyword}\textbf}, 
        identifierstyle = {\ttfamily}, 
        commentstyle = {\ttfamily\color{listcomment}\textit}, 
        stringstyle = {\ttfamily}, 
        showstringspaces = false, 
        showtabs = false, 
        numbers = left, 
        numbersep = 6pt, 
        numberstyle={\ttfamily\color{listnumbers}}, 
        tabsize = 2, 
        floatplacement=!h 
        } 
}

% Java Definition

\newcommand{\lstsetJava}[1]
{ 
\lstset{caption={#1},
	language=Java, 
	breaklines=true,
	frame = tb, 
        framerule = 0.25pt, 
        float, 
        fontadjust, 
        backgroundcolor={\color{listlightgray}}, 
        basicstyle = {\ttfamily\footnotesize}, 
        keywordstyle = {\ttfamily\color{listkeyword}\textbf}, 
        identifierstyle = {\ttfamily}, 
        commentstyle = {\ttfamily\color{listcomment}\textit}, 
        stringstyle = {\ttfamily}, 
        showstringspaces = false, 
        showtabs = false, 
        numbers = left, 
        numbersep = 6pt, 
        numberstyle={\ttfamily\color{listnumbers}}, 
        tabsize = 2, 
        floatplacement=!h 
        } 
}

% Ruby Definition

\newcommand{\lstsetRuby}[1]
{ 
\lstset{caption={#1}, 
	language=Ruby, 
	breaklines=true,
	frame = tb, 
        framerule = 0.25pt, 
        float, 
        fontadjust, 
        backgroundcolor={\color{listlightgray}}, 
        basicstyle = {\ttfamily\footnotesize}, 
        keywordstyle = {\ttfamily\color{listkeyword}\textbf}, 
        identifierstyle = {\ttfamily}, 
        commentstyle = {\ttfamily\color{listcomment}\textit}, 
        stringstyle = {\ttfamily}, 
        showstringspaces = false, 
        showtabs = false, 
        numbers = left, 
        numbersep = 6pt, 
        numberstyle={\ttfamily\color{listnumbers}}, 
        tabsize = 2, 
        floatplacement=!h 
        } 
}

% Python Definition

\newcommand{\lstsetPython}[1]
{ 
\lstset{caption={#1}, 
	language=Python,
	breaklines=true,
	frame = tb, 
        framerule = 0.25pt, 
        float, 
        fontadjust, 
        backgroundcolor={\color{listlightgray}}, 
        basicstyle = {\ttfamily\footnotesize}, 
        keywordstyle = {\ttfamily\color{listkeyword}\textbf}, 
        identifierstyle = {\ttfamily}, 
        commentstyle = {\ttfamily\color{listcomment}\textit}, 
        stringstyle = {\ttfamily}, 
        showstringspaces = false, 
        showtabs = false, 
        numbers = left, 
        numbersep = 6pt, 
        numberstyle={\ttfamily\color{listnumbers}}, 
        tabsize = 2, 
        floatplacement=!h 
        } 
}

% JavaScript Definition

\lstdefinelanguage{JavaScript}{
    keywords={attributes, class, classend, do, empty, endif, endwhile, fail, function, functionend, if, implements, in, inherit, inout, not, of, operations, out, return, set, then, types, while, use},
    keywordstyle=\color{blue}\bfseries,
    ndkeywords={},
    ndkeywordstyle=\color{yellow}\bfseries,
    identifierstyle=\color{black},
    sensitive=false,
    comment=[l]{//},
    commentstyle=\color{green}\ttfamily,
    stringstyle=\color{red}\ttfamily
 }

\newcommand{\lstsetJavaScript}[1]
{ 
\lstset{caption={#1}, 
	language=JavaScript,
	breaklines=true,
	frame = tb, 
        framerule = 0.25pt, 
        float, 
        fontadjust, 
        backgroundcolor={\color{listlightgray}}, 
        basicstyle = {\ttfamily\footnotesize}, 
        keywordstyle = {\ttfamily\color{listkeyword}\textbf}, 
        identifierstyle = {\ttfamily}, 
        commentstyle = {\ttfamily\color{listcomment}\textit}, 
        stringstyle = {\ttfamily}, 
        showstringspaces = false, 
        showtabs = false, 
        numbers = left, 
        numbersep = 6pt, 
        numberstyle={\ttfamily\color{listnumbers}}, 
        tabsize = 2, 
        floatplacement=!h 
        } 
}

% Examples 

\begin{comment}
\begin{center} 
	\lstsetPython{Python - Class Example}
	\lstinputlisting
	{path/file.ext} 
\end{center}	

\lstsetCpp
\begin{lstlisting} 
unsigned int lifeUniverseAndEverything = 42; 

public class Animal {
	public static void main(String[] args) {
		System.out.println("Buenas, soy un aprendiz de Java!");
	}

	private Location loc;
	private double energyReserves;
 
	boolean isHungry() {
		return energyReserves < 2.5;
	}
	void eat(Food f) {
		// Consume food
		energyReserves += f.getCalories();
	 }
	void moveTo(Location l) {
		// Move to new location
		loc = l;
	}
}
\end{lstlisting}
\end{comment}
