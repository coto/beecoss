\section{Class}

Informalmente, un objeto representa una entidad (Física, Conceptual o Software) del mundo real, además posee (según Booch) Estado, Comportamiento e Identidad

La clase es el tipo del objeto, es decir, es una descripción de un grupo de objetos con propiedades en común (atributos), comportamiento similar (operaciones), misma forma de relacionarse con otros objetos (relaciones) y una semántica común 

* Atributos: representan los estados del objeto
* Metodos: representan el comportamiento del objeto.

El ideal, purista, es que los metodos sean publicos y todos los atributos sean privados.

Todos los objetos necesitan de un constructor, que tiene el mismo nombre de la clase, el cual es un metodo que reserva memoria. 

Usuarios coto = new User();

Puede existir un constructor que reciba parametros.

Usuario coto = new User("coto", "1234");

Java y c\# inicializan los estados del objeto con valores nulos, vacios (0 en caso de numerico) o = en caso de numéricos.

-----------------------------------o---------------------

Sobrecarga de metodos "overloading" != sobreescritura de metodos "overwriting" (polimorfismo)

"overloading" no es propio de la POO, y consiste en una clase con varios métodos con el mismo nombre pero diferente “firma”.

"overwriting" La clase base define métodos, los cuales pueden ser reescritos por clases que heredan de ella. 

User();
User(string);
User(string, int, string);

public (modificador de acceso) -

Modificador de acceso: [public, private, protected] indica que un metodo o atributo puede ser accedido desde otra clase o solo internamente (private), es muy mala práctica declarar estados privados.

Atributos - Contructores - Metodos, es el orden adecuado dentro de una clase.

%### Properties (c\#) | Getters & Setters (java) | Accesor (Ruby) ###

Metodos para acceder a los atributos de los objetos


Atributos:metodos:sobrecarga: constructos: porperties:modificadores de accesos (TODO)


