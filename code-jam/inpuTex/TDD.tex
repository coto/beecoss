\section{TDD}
\newcounter{Lcount}

\textbf{Test-driven development} (TDD) is a software development process that relies on the repetition of a very short development cycle with these steps:

\begin{enumerate}
\item the developer writes a failing automated test case that defines a desired improvement or new function.
\item Run all tests and see if the new one fails.
\item Produces code to pass that test.
\item Run the automated tests and see them succeed.
\item Refactors the new code to acceptable standards. 
\end{enumerate}


Steps to create a unit test:


\begin{enumerate}
\item Establecer los datos de prueba y retorno
\item usar los datos de prueba para ejecutar el codigo que se esta probando
\item Generar las pruebas necesarias
\end{enumerate}


Unit tests are so named because they each test one unit of code. 

\begin{center} 
	\lstsetCpp{Example of Unit Test using C\#}
\begin{lstlisting} 
[Test()]
//Usada solo para chequear el tipo de la excepcion.
[ExpectedException(typeof(NullReferenceException))]
// IF "expression" returns null AND  "EmptyDataTemplate" is not null THEN 
public void List_ExpressionReturnsNullAndEmptyDataTemplateisNotNull_ReturnsEmptyString() {
	ValueToTest = .....
}


//Usada para chequear otras validaciones
public void List_Conditions_ReturnsValue() {
	ValueToTest = .....
	Assert(String.Empty, ValueToTest);
}
\end{lstlisting}
\end{center}





