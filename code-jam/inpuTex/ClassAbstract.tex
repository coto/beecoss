\section{Class Abstracts}

@@@ Pilares de la Orientación a Objetos @@@

** Abstracción **

La abstraccion es un metodo por el cual abstraemos, vale la redundancia, una determinada entidad de la realidad sus caracteristicas y funciones que desempeñan, estos son representados en clases por medio de atributos y metodos de dicha clase. 

Ejemplo: Un ejemplo sencillo para comprender este concepto seria la abstraccion de un Automovil. Aca vamos a sacar de estas entidad sus caracteristicas por ejemplo: color, año de fabricacion, modelo, etc. Y ahora sacamos sus metodos o funciones tipicas de esta entidad como por ejemplo: frenar, encender, etc. 

A esto se le llama abstracción.


** Encapsulamiento ** 

Atributos deben ser privados para otros objetos, exponiendolos solo a través del comportamiento definido a través de miembros públicos.

Util para el control/validación y respuesta ante cambios.

** Relaciones **

Los objetos contribuyen en el comportamiento de un sistema  colaborando entre si  a través de sus relaciones.

Una Relación de asociación es una conexión entre dos clases que representa una comunicación (e.g. Una Persona es Dueña de un Vehículo)

Una Relación de  agregación es una forma especial de asociación donde un todo se relaciona con sus partes (e.g. Una Puerta es una parte de un Vehículo)

** Herencia **

Es un tipo de relación entre clases en la cual una clase comparte la estructura y comportamiento definido en otra clase (Grady Booch)

@@@ Conceptos del Diseño Orientado a Objetos  @@@

** Interfaces **

Recurso de diseño soportado por los lenguajes orientados a objetos que permite definir comportamiento.

La implementación de una interfaz es un contrato que obliga a la clase a implementar todos los métodos definidos en la interfaz.


** Polimorfismo **

Es la propiedad que tienen los objetos de permitir invocar genéricamente un comportamiento (método) cuya implementación será delegada al objeto correspondiente recién en tiempo de ejecución.


