% Created with $ pdflatex POO.tex

\documentclass[14pt]{article}
\usepackage{hyperref} % Make links
\usepackage{listings} % Write code
\usepackage{verbatim} % Multiline comments
\usepackage{color}

	\definecolor{listcomment}{rgb}{0.0,0.5,0.0}
	\definecolor{listkeyword}{rgb}{0.0,0.0,0.5}
	\definecolor{listnumbers}{gray}{0.65}
	\definecolor{listlightgray}{gray}{0.955}
	\definecolor{listwhite}{gray}{1.0}

% C++ Definition

\newcommand{\lstsetCpp}[1]
{ 
\lstset{caption={#1}, 
	language=[ANSI]C++,
	breaklines=true, 
        frame = tb, 
        framerule = 0.25pt, 
        float, 
        fontadjust, 
        backgroundcolor={\color{listlightgray}}, 
        basicstyle = {\ttfamily\footnotesize}, 
        keywordstyle = {\ttfamily\color{listkeyword}\textbf}, 
        identifierstyle = {\ttfamily}, 
        commentstyle = {\ttfamily\color{listcomment}\textit}, 
        stringstyle = {\ttfamily}, 
        showstringspaces = false, 
        showtabs = false, 
        numbers = left, 
        numbersep = 6pt, 
        numberstyle={\ttfamily\color{listnumbers}}, 
        tabsize = 2, 
        floatplacement=!h 
        } 
}

% Java Definition

\newcommand{\lstsetJava}[1]
{ 
\lstset{caption={#1},
	language=Java, 
	breaklines=true,
	frame = tb, 
        framerule = 0.25pt, 
        float, 
        fontadjust, 
        backgroundcolor={\color{listlightgray}}, 
        basicstyle = {\ttfamily\footnotesize}, 
        keywordstyle = {\ttfamily\color{listkeyword}\textbf}, 
        identifierstyle = {\ttfamily}, 
        commentstyle = {\ttfamily\color{listcomment}\textit}, 
        stringstyle = {\ttfamily}, 
        showstringspaces = false, 
        showtabs = false, 
        numbers = left, 
        numbersep = 6pt, 
        numberstyle={\ttfamily\color{listnumbers}}, 
        tabsize = 2, 
        floatplacement=!h 
        } 
}

% Ruby Definition

\newcommand{\lstsetRuby}[1]
{ 
\lstset{caption={#1}, 
	language=Ruby, 
	breaklines=true,
	frame = tb, 
        framerule = 0.25pt, 
        float, 
        fontadjust, 
        backgroundcolor={\color{listlightgray}}, 
        basicstyle = {\ttfamily\footnotesize}, 
        keywordstyle = {\ttfamily\color{listkeyword}\textbf}, 
        identifierstyle = {\ttfamily}, 
        commentstyle = {\ttfamily\color{listcomment}\textit}, 
        stringstyle = {\ttfamily}, 
        showstringspaces = false, 
        showtabs = false, 
        numbers = left, 
        numbersep = 6pt, 
        numberstyle={\ttfamily\color{listnumbers}}, 
        tabsize = 2, 
        floatplacement=!h 
        } 
}

% Python Definition

\newcommand{\lstsetPython}[1]
{ 
\lstset{caption={#1}, 
	language=Python,
	breaklines=true,
	frame = tb, 
        framerule = 0.25pt, 
        float, 
        fontadjust, 
        backgroundcolor={\color{listlightgray}}, 
        basicstyle = {\ttfamily\footnotesize}, 
        keywordstyle = {\ttfamily\color{listkeyword}\textbf}, 
        identifierstyle = {\ttfamily}, 
        commentstyle = {\ttfamily\color{listcomment}\textit}, 
        stringstyle = {\ttfamily}, 
        showstringspaces = false, 
        showtabs = false, 
        numbers = left, 
        numbersep = 6pt, 
        numberstyle={\ttfamily\color{listnumbers}}, 
        tabsize = 2, 
        floatplacement=!h 
        } 
}

% JavaScript Definition

\lstdefinelanguage{JavaScript}{
    keywords={attributes, class, classend, do, empty, endif, endwhile, fail, function, functionend, if, implements, in, inherit, inout, not, of, operations, out, return, set, then, types, while, use},
    keywordstyle=\color{blue}\bfseries,
    ndkeywords={},
    ndkeywordstyle=\color{yellow}\bfseries,
    identifierstyle=\color{black},
    sensitive=false,
    comment=[l]{//},
    commentstyle=\color{green}\ttfamily,
    stringstyle=\color{red}\ttfamily
 }

\newcommand{\lstsetJavaScript}[1]
{ 
\lstset{caption={#1}, 
	language=JavaScript,
	breaklines=true,
	frame = tb, 
        framerule = 0.25pt, 
        float, 
        fontadjust, 
        backgroundcolor={\color{listlightgray}}, 
        basicstyle = {\ttfamily\footnotesize}, 
        keywordstyle = {\ttfamily\color{listkeyword}\textbf}, 
        identifierstyle = {\ttfamily}, 
        commentstyle = {\ttfamily\color{listcomment}\textit}, 
        stringstyle = {\ttfamily}, 
        showstringspaces = false, 
        showtabs = false, 
        numbers = left, 
        numbersep = 6pt, 
        numberstyle={\ttfamily\color{listnumbers}}, 
        tabsize = 2, 
        floatplacement=!h 
        } 
}

% Examples 

\begin{comment}
\begin{center} 
	\lstsetPython{Python - Class Example}
	\lstinputlisting
	{path/file.ext} 
\end{center}	

\lstsetCpp
\begin{lstlisting} 
unsigned int lifeUniverseAndEverything = 42; 

public class Animal {
	public static void main(String[] args) {
		System.out.println("Buenas, soy un aprendiz de Java!");
	}

	private Location loc;
	private double energyReserves;
 
	boolean isHungry() {
		return energyReserves < 2.5;
	}
	void eat(Food f) {
		// Consume food
		energyReserves += f.getCalories();
	 }
	void moveTo(Location l) {
		// Move to new location
		loc = l;
	}
}
\end{lstlisting}
\end{comment}


\title{Object-Oriented Programming}
\author{\href{http://twitter.com/coto}{Rodrigo Augosto}, \href{http://twitter.com/janogonzalez}{Alejandro Gonzalez} }
\date{November 16, 2010}

\begin{document}
\maketitle						% automatic title!

This document has been prepared to guide learners of 
Object-Oriented Programming (\textbf{OOP}).
 
We used \LaTeX{} to make this document and there are
examples to each term involved using Python, Ruby, C\#, Java and JavaScript

\tableofcontents

\section{ProgrammingFundamentals}


\begin{itemize}
\item Existen dos tipos de Software: De sistema y de Aplicación.
\item Lenguajes de programación pueden ser: De máquina, Bajo nivel y Alto Nivel.
\item Tipo de Lenguajes:
	\begin{itemize}
		\item Estructurados (C, Pascal, Basic)
		\item Orientados a Objetos (C\#, VB.NET, Smalltalk, Java)
		\item Declarativos (Prolog)
		\item Funcionales (CAML)
		\item Orientados a Aspectos
		\item Híbridos (Lisp, Visual Basic)
	\end{itemize}
\item Las Sentencias, describen acciones algorítmicas que pueden ser ejecutadas, clasificadas en:
	\begin{itemize}
		\item Ejecutables / No ejecutables
		\item Simples / Estructuradas
	\end{itemize}
\item Una expresión es un conjunto de datos unidos por operadores que tiene un único resultado (a = ((2+6) / 8) * 3).
\item Las estructuras de control permiten alterar el orden del flujo de control
	\begin{itemize}
		\item Estructuras de Control Selectivas; IF, CASE
		\item Estructuras de Control Repetitivas: FOR, WHILE
	\end{itemize}
\item Existen diversos tipos de operadores:
	\begin{itemize}
		\item Aritméticos: suma, resta, multiplicación, etc.
		\item De relación: igual, mayor, menor, distinto, etc.
		\item Lógicos: and, or, not, etc.
	\end{itemize}
\item Alcance o tipo de miebros se refiere a los campos, propiedades, métodos, eventos, clases anidadas, etc
	\begin{itemize}
		\item Public
		\item Private
		\item Internal
		\item Protected
		\item Protected Internal
	\end{itemize}
\item Las comillas dobles ("") delimitan strings y las comillas simples ('') delimintan caracteres.
\end{itemize}

Bibliotecas
Desarrollo -> Programa Fuente
Compilación -> Programa Objeto
Link-Edición -> Programa Ejecutable


??? Métodos estáticos: no requieren de una instancia para ser invocados. Se los llama métodos “de clase” ???

??? Namespace de una clase ???



\section{Class}

Informalmente, un objeto representa una entidad (Física, Conceptual o Software) del mundo real, además posee (según Booch) Estado, Comportamiento e Identidad

La clase es el tipo del objeto, es decir, es una descripción de un grupo de objetos con propiedades en común (atributos), comportamiento similar (operaciones), misma forma de relacionarse con otros objetos (relaciones) y una semántica común 

* Atributos: representan los estados del objeto
* Metodos: representan el comportamiento del objeto.

El ideal, purista, es que los metodos sean publicos y todos los atributos sean privados.

Todos los objetos necesitan de un constructor, que tiene el mismo nombre de la clase, el cual es un metodo que reserva memoria. 

Usuarios coto = new User();

Puede existir un constructor que reciba parametros.

Usuario coto = new User("coto", "1234");

Java y c\# inicializan los estados del objeto con valores nulos, vacios (0 en caso de numerico) o = en caso de numéricos.

-----------------------------------o---------------------

Sobrecarga de metodos "overloading" != sobreescritura de metodos "overwriting" (polimorfismo)

"overloading" no es propio de la POO, y consiste en una clase con varios métodos con el mismo nombre pero diferente “firma”.

"overwriting" La clase base define métodos, los cuales pueden ser reescritos por clases que heredan de ella. 

User();
User(string);
User(string, int, string);

public (modificador de acceso) -

Modificador de acceso: [public, private, protected] indica que un metodo o atributo puede ser accedido desde otra clase o solo internamente (private), es muy mala práctica declarar estados privados.

Atributos - Contructores - Metodos, es el orden adecuado dentro de una clase.

%### Properties (c\#) | Getters & Setters (java) | Accesor (Ruby) ###

Metodos para acceder a los atributos de los objetos


Atributos:metodos:sobrecarga: constructos: porperties:modificadores de accesos (TODO)



	\begin{center} 
		\lstsetJava{Java - Class Example}
		\lstinputlisting
		{inpuTex_java/ClassExample.java} 
	\end{center}

	\begin{center} 
		\lstsetCpp{C\# - Class Example}
		\lstinputlisting
		{inpuTex_csharp/ClassExample.cs} 
	\end{center}

	\begin{center} 
		\lstsetPython{Python - Class Example}
		\lstinputlisting
		{inpuTex_python/ClassExample.py} 
	\end{center}

	\begin{center} 
		\lstsetRuby{Ruby - Class Example}
		\lstinputlisting
		{inpuTex_ruby/ClassExample.rb} 
	\end{center}

	\begin{center} 
		\lstsetJavaScript{JavaScript - Class Example}
		\lstinputlisting
		{inpuTex_javascript/ClassExample.js} 
	\end{center}

\subsection{Encapsulation}

Obj: No modificar estados internos de un objeto de forma directa, sino a través de lods metodos expuestos.

el objeto tiene un estado interno que este representado por cada uno de sus atributos, el cual no puede ser cambiado forma directa, a mneos que se exponga un metodo de tipo publico que lo haga.


\subsection{Inheritance}

Obj: Este es un concepto fundamental para la POO y para el lenguaje java, ya que con este concepto significa que vamos a poder reutilizar codigo. Un ejemplo seria una clase Figura Geometrica, que tiene funciones como el calculo de su perimetro y de su area, y tiene como sub clase la clase Cuadrado, que era sus metodos de la clase Figura Geometrica que son en este caso perimetro y area, aca aprovechamos la reutilizacion de codigo. Pero en java no existe la herencia multiple como en otros programas como C/C++, aca es solo herencia simple, pero en java existe algo que simula esta herencia multiple que son las llamadas interfaces que posteriormente vamos a estudiar.

\subsection{Polymorphism}

Obj: Dado un mensaje despues de una instancia a una/varias subclases, todas deberán responder de forma diferente.

Este concepto se base en que podemos utilizar varios metodos con el mismo nombre y con diferente funcionalidad. Por ejemplo de mi clase Vehiculos tenemos el metodo frenar, y tenemos sus sub clases Automovil y Motocibleta, ambos tienen el metodo frenar pero cada uno tiene una definicion diferente para cada clase. A esto se le denomina polimorfismo, mas adelante lo vamos a ver en las sobrecargas de funciones y redefiniciones de metodos en la herencia.


\input{inpuTex/Casting}
\section{Class Abstracts}

@@@ Pilares de la Orientación a Objetos @@@

** Abstracción **

La abstraccion es un metodo por el cual abstraemos, vale la redundancia, una determinada entidad de la realidad sus caracteristicas y funciones que desempeñan, estos son representados en clases por medio de atributos y metodos de dicha clase. 

Ejemplo: Un ejemplo sencillo para comprender este concepto seria la abstraccion de un Automovil. Aca vamos a sacar de estas entidad sus caracteristicas por ejemplo: color, año de fabricacion, modelo, etc. Y ahora sacamos sus metodos o funciones tipicas de esta entidad como por ejemplo: frenar, encender, etc. 

A esto se le llama abstracción.


** Encapsulamiento ** 

Atributos deben ser privados para otros objetos, exponiendolos solo a través del comportamiento definido a través de miembros públicos.

Util para el control/validación y respuesta ante cambios.

** Relaciones **

Los objetos contribuyen en el comportamiento de un sistema  colaborando entre si  a través de sus relaciones.

Una Relación de asociación es una conexión entre dos clases que representa una comunicación (e.g. Una Persona es Dueña de un Vehículo)

Una Relación de  agregación es una forma especial de asociación donde un todo se relaciona con sus partes (e.g. Una Puerta es una parte de un Vehículo)

** Herencia **

Es un tipo de relación entre clases en la cual una clase comparte la estructura y comportamiento definido en otra clase (Grady Booch)

@@@ Conceptos del Diseño Orientado a Objetos  @@@

** Interfaces **

Recurso de diseño soportado por los lenguajes orientados a objetos que permite definir comportamiento.

La implementación de una interfaz es un contrato que obliga a la clase a implementar todos los métodos definidos en la interfaz.


** Polimorfismo **

Es la propiedad que tienen los objetos de permitir invocar genéricamente un comportamiento (método) cuya implementación será delegada al objeto correspondiente recién en tiempo de ejecución.



\input{inpuTex/Delegates}
\input{inpuTex/Enumeration}
\input{inpuTex/Generics}
\input{inpuTex/Interface}
\section{Lambda}

TODO

Cuantas formas existen de expresar un puntero a una funcion
\begin{enumerate}
\item delegado on the fly
\item usando un metod ya existente
\item usando expresiones \textbf{lambda}
\end{enumerate}

\section{TDD}
\newcounter{Lcount}

\textbf{Test-driven development} (TDD) is a software development process that relies on the repetition of a very short development cycle with these steps:

\begin{enumerate}
\item the developer writes a failing automated test case that defines a desired improvement or new function.
\item Run all tests and see if the new one fails.
\item Produces code to pass that test.
\item Run the automated tests and see them succeed.
\item Refactors the new code to acceptable standards. 
\end{enumerate}


Steps to create a unit test:


\begin{enumerate}
\item Establecer los datos de prueba y retorno
\item usar los datos de prueba para ejecutar el codigo que se esta probando
\item Generar las pruebas necesarias
\end{enumerate}


Unit tests are so named because they each test one unit of code. 

\begin{center} 
	\lstsetCpp{Example of Unit Test using C\#}
\begin{lstlisting} 
[Test()]
//Usada solo para chequear el tipo de la excepcion.
[ExpectedException(typeof(NullReferenceException))]
// IF "expression" returns null AND  "EmptyDataTemplate" is not null THEN 
public void List_ExpressionReturnsNullAndEmptyDataTemplateisNotNull_ReturnsEmptyString() {
	ValueToTest = .....
}


//Usada para chequear otras validaciones
public void List_Conditions_ReturnsValue() {
	ValueToTest = .....
	Assert(String.Empty, ValueToTest);
}
\end{lstlisting}
\end{center}









\end{document}             % End of document.
