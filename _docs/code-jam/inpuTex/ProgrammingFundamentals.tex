\section{ProgrammingFundamentals}


\begin{itemize}
\item Existen dos tipos de Software: De sistema y de Aplicación.
\item Lenguajes de programación pueden ser: De máquina, Bajo nivel y Alto Nivel.
\item Tipo de Lenguajes:
	\begin{itemize}
		\item Estructurados (C, Pascal, Basic)
		\item Orientados a Objetos (C\#, VB.NET, Smalltalk, Java)
		\item Declarativos (Prolog)
		\item Funcionales (CAML)
		\item Orientados a Aspectos
		\item Híbridos (Lisp, Visual Basic)
	\end{itemize}
\item Las Sentencias, describen acciones algorítmicas que pueden ser ejecutadas, clasificadas en:
	\begin{itemize}
		\item Ejecutables / No ejecutables
		\item Simples / Estructuradas
	\end{itemize}
\item Una expresión es un conjunto de datos unidos por operadores que tiene un único resultado (a = ((2+6) / 8) * 3).
\item Las estructuras de control permiten alterar el orden del flujo de control
	\begin{itemize}
		\item Estructuras de Control Selectivas; IF, CASE
		\item Estructuras de Control Repetitivas: FOR, WHILE
	\end{itemize}
\item Existen diversos tipos de operadores:
	\begin{itemize}
		\item Aritméticos: suma, resta, multiplicación, etc.
		\item De relación: igual, mayor, menor, distinto, etc.
		\item Lógicos: and, or, not, etc.
	\end{itemize}
\item Alcance o tipo de miebros se refiere a los campos, propiedades, métodos, eventos, clases anidadas, etc
	\begin{itemize}
		\item Public
		\item Private
		\item Internal
		\item Protected
		\item Protected Internal
	\end{itemize}
\item Las comillas dobles ("") delimitan strings y las comillas simples ('') delimintan caracteres.
\end{itemize}

Bibliotecas
Desarrollo -> Programa Fuente
Compilación -> Programa Objeto
Link-Edición -> Programa Ejecutable


??? Métodos estáticos: no requieren de una instancia para ser invocados. Se los llama métodos “de clase” ???

??? Namespace de una clase ???
